% Template for Carleton comps papers
% Author: Andrew Gainer-Dewar, 2013
\documentclass[twoside]{memoir}
\usepackage{cccomps}

% The Latin Modern font is a modernized replacement for the classic
% Computer Modern. Feel free to replace this with a different font package.
\usepackage{lmodern}

% Load in biblatex
% To use a different bibliography style, just change "numeric" to
% your preferred style (mla for MLA style, alphabetic for Author-Year
% style, etc.) There are a lot of options; check the BibLaTeX documentation.
\usepackage[backend=biber,style=numeric]{biblatex}
% Select the bibliography file
\addbibresource{sources-example.bib}

% We frequently use the monospace style to represent code.
% Rather than use the \texttt{} command all the time (which is
% unsemantic and ugly to read), we define a new semantic command
% \code{} which does this for us.
\newcommand{\code}[1]{\texttt{#1}}

% We'll use the TikZ diagramming language for an example
\usepackage{tikz,tikz-cd}
\usetikzlibrary{matrix, calc, arrows}
\tikzset{node distance = 1.5cm, auto}
\DeclareMathOperator{\coker}{coker}

% The listings package handles typesetting nice code listings.
% We set it up here with some nice options.
\RequirePackage{listings}
\RequirePackage{xcolor}
\lstset{
  basicstyle=\footnotesize\ttfamily,
  commentstyle=\color{gray},
  numbers=left,
  breaklines=true,
  breakatwhitespace=true,
  numberstyle=\footnotesize,
  stepnumber=2,
  showstringspaces=false,
}
\crefname{lstlisting}{listing}{listings}
\Crefname{lstlisting}{Listing}{Listings}

\title{Example comps project}
\author{Lori T.~Jeremiah \and Ray B.~Langstrom} % The \and command takes care of putting the names side-by-side.
\advisor{M.~Holmes}
\dept{Mathemagical Inquiry}
\date{February 31, 2014}

\begin{document}
% First, we go into "front matter" mode.
% Among other things, this gives us Roman page numbers.
\frontmatter

% We tell LaTeX to make a title page.
\maketitle

% Then we do some front-matter business.
\chapter{Dedication}
We dedicate this project to Nicolas Bourbaki, whose mathematical acumen and personal warmth were invaluable resources to us.

\begin{abstract}
  The senior integrative exercise, or \enquote{Comps}, is part of every Carleton major and a graduation requirement.
  The \code{cccomps} style and template provide a structure for writing Comps papers using the \LaTeX{} computer typesetting system.
\end{abstract}

% A Table of Contents is important!
\tableofcontents

% Include the list of figures and list of tables only if you actually *have*
% figures and tables! (The * after each indicates that it should not be included
% in the table of contents.)
%\listoffigures*
%\listoftables*

% Next, we go into "main matter" mode.
% This resets the page numbers and uses Arabic numerals.
\mainmatter

\chapter{Using the Comps template}
\label{s:usingtemplate}
So it's time to write up your Comps project!
The Carleton \LaTeX{} Comps template is provided to help you do this efficiently and with a minimum of fuss.
Good luck, and have fun!

\section{Getting started with \LaTeX{}}
\label{s:gettingstarted}
If you aren't already familiar with the \LaTeX{} computer typesetting system, this isn't the right document for you.
Check out the resources available to you through the Carleton \LaTeX{} workshop, in the following order:
\begin{enumerate}
\item
  \enquote{Getting started with \LaTeX{}} is a quick-start guide designed to bring you from a state of zero knowledge to the point where you can write a basic document.

\item
  The \enquote{Math typesetting guide} will teach you how to use some of \LaTeX{}'s impressive math typesetting facilities.
  This will be important if you're writing a comps paper in a mathematical discipline.

\item
  The \enquote{Text typesetting guide} will teach you how to use \LaTeX{} for formatting and typesetting text.
  This includes typeface formatting, sectioning, lists (like this one), internal cross-references, and other concerns.

\item
  The \enquote{Citation guide} will teach you how to use the Bib\LaTeX{} bibliography processor to simplify your citation process.
  Your Comps paper should include \emph{lots} of citations to the literature, so this will be very valuable!
\end{enumerate}

\section{Getting started with the template}
\label{s:template}
Once you know your way around \LaTeX{}, you'll be ready to get started with the \code{cccomps} package.
First, you'll need to download it.
Be sure that \code{cccomps.sty} (the \enquote{style file}), \code{cccomps.tex} (the \enquote{template}), and \code{sources.bib} (the \enquote{bibliographic database}) end up in the same directory so \LaTeX{} will know how to find them.

Once you have these files, open \code{cccomps.tex} in your preferred \LaTeX{} editor.
Read it through, including the comments.
Once you've gotten a sense of the structure of the file, you can start to add in your own content.

\section{The structure of the template}
\label{s:structure}
The Comps template can be decomposed into several key components.

\subsection{The preamble}
\label{s:preamble}
The first component of the template is the \enquote{preamble}, which sets up the \LaTeX{} environment and defines important properties of the document as a whole.
The preamble of the Comps template appears in \cref{lst:cccompspreamble}.
For editing purposes, the most important part of the preamble is the final five lines, which define the metadata for the document.
Go ahead and replace their contents with your title, name, and so on.

\lstinputlisting[float=p,frame=single, caption=The Carleton Comps template preamble, label=lst:cccompspreamble, language=TeX, firstline=1, firstnumber=1, lastline=22]{cccomps.tex}

The preamble is also the place where you should load any packages you need to, for example, change the font used in your paper or use the TikZ diagramming language.

\subsection{The front matter}
\label{s:frontmatter}
After the preamble comes the \enquote{front matter}, which is the part of the document before the body text.
The front matter of the Comps template appears in \cref{lst:cccompsfront}.

\lstinputlisting[float=p,frame=single, caption=The Carleton Comps template front matter, label=lst:cccompsfront, language=TeX, firstline=24, firstnumber=24, lastline=48]{cccomps.tex}

If you wish to add a dedication, do so here; otherwise, delete the \code{\textbackslash{}chapter\{Dedication\}} line and the one which follows it.
You can also add an acknowledgments section or any other front matter by adding a new \code{\textbackslash{}chapter\{\}}.

You should also write an abstract for your paper.
In keeping with the conventions of mathematical writing, this abstract should be no longer than three hundred words, and it should outline the main results and methods of your paper.
It should not contain any references or citations.

Finally, if you wish for your paper to have a list of figures or a list of tables, uncomment the appropriate line by removing the \code{\%}.

\subsection{The main matter}
\label{s:mainmatter}
After the front matter comes the \enquote{main matter}, which is where you'll write the body of your paper.
The main matter of the Comps template appears in \cref{lst:cccompsmain}.

\lstinputlisting[float=p,frame=single, caption=The Carleton Comps template main matter, label=lst:cccompsmain, language=TeX, firstline=50, firstnumber=50, lastline=71]{cccomps.tex}

Your paper should be organized into \code{chapter}s, each of which may have \code{section}s, \code{subsection}s, and even \code{subsubsection}s.
(If it's really long, you may split your document up into \code{part}s, each of which has several \code{chapter}s.)
See the \enquote{Text formatting guide} for more information about document structuring and internal references.

If you wish to have any appendices, use the \code{\textbackslash{}appendix} command to switch \LaTeX{} into appendix mode.
Any \code{chapter} which you write after calling \code{\textbackslash{}appendix} will automatically be an appendix.
If you do \emph{not} wish to have appendices, just remove the line that calls \code{\textbackslash{}appendix}.

\subsection{The back matter}
\label{s:backmatter}
After the main matter comes the \enquote{back matter}, the final section of the document.
The back matter of the Comps template appears in \cref{lst:cccompsback}.

\lstinputlisting[float=p,frame=single, caption=The Carleton Comps template back matter, label=lst:cccompsback, language=TeX, firstline=73, firstnumber=73, lastline=76]{cccomps.tex}

You should not need to modify anything in the back matter.
The \code{\textbackslash{}printbibliography} command tells Bib\LaTeX{} where to put your bibliography, but this is configured elsewhere.

\section{\LaTeX{} fanciness}
\label{s:fanciness}
The code shown in \cref{s:structure} is all you need to write a Carleton Comps paper.
Notice, for example, that there is no code to specify how the page headers, chapter titles, or title page are formatted, and nothing is done to make chapters start on front-facing pages---this is all handled automatically by the style and class files!

In addition, thanks to \LaTeX{}, you can easily include beautiful mathematics, professional-looking tables, and other delights in your paper.
In \cref{s:example}, we'll demonstrate some of these techniques and their results.

\chapter{Examples of \LaTeX{} typography}
\label{s:example}
\section{Mathematical typesetting}
\label{s:math}
\LaTeX{} is extremely good at typesetting mathematics.
The \enquote{Math typesetting guide} has lots of information about how to achieve great results with many kinds of math; we present here some examples to illustrate the point.

Here is an important definition from \cite[\S 4.1]{hend}:
\begin{subequations}
  \label{eq:gcidef}
  \begin{align}
    Z_{\mathcal{F}}^{\Gamma} (\gamma) &= \sum_{\substack{n \geq 0 \\ \sigma \in \mathfrak{S}_{n}}} \frac{1}{n!} \abs*{ \operatorname{fix} \mathcal{F} \sbrac{ \sigma \cdot \gamma } } p_{1}^{\sigma_{1}} p_{2}^{\sigma_{@}} \dots p_{n}^{\sigma_{n}} \label{eq:gcidefperm} \\
    &= \sum_{\substack{n \geq 0 \\ \lambda \vdash n}} \frac{\abs*{ \operatorname{fix} \mathcal{F} \sbrac{\sigma(\lambda) \cdot \gamma} }}{z_{\lambda}} p_{1}^{\lambda_{1}} p_{2}^{\lambda_{2}} \dots p_{n}^{\lambda_{n}}. \label{eq:gcidefpart}
  \end{align}
\end{subequations}

We would be remiss to omit Stokes' theorem from any discussion of elegant typography:
\begin{equation}
  \label{eq:stokes}
  \int_{\partial \Omega} \omega = \int_{\omega} \operatorname{d} \omega.
\end{equation}

If we allow ourselves the luxury of drawing some commutative diagrams, we can obtain the following important result:
\begin{lemma*}[Snake lemma]
  Suppose that the following diagram commutes in some abelian category:
  \begin{center}
    \begin{tikzpicture}
      \node (0l) {$0$};
      \node (A') [right of=0l] {$A'$};
      \node (B') [right of=A'] {$B'$};
      \node (C') [right of=B'] {$C'$};
      \node (A) [above of=A'] {$A$};
      \node (B) [above of=B'] {$B$};
      \node (C) [above of=C'] {$C$};
      \node (0r) [right of=C] {$0$};

      \draw [->] (0l) to (A');
      \draw [->] (A') to node {$f'$} (B');
      \draw [->] (B') to node {$g'$} (C');
      \draw [->] (A) to node {$a$} (A');
      \draw [->] (A) to node {$f$} (B);
      \draw [->] (B) to node {$b$} (B');
      \draw [->] (B) to node {$g$} (C);
      \draw [->] (C) to node {$c$} (C');
      \draw [->] (C) to (0r);
    \end{tikzpicture}
  \end{center}
  Then there is a map $d: \ker c \to \coker a$ such that the following diagram commutes and all sequences are exact:
    \begin{center}
    \begin{tikzpicture}
      \matrix[matrix of math nodes,column sep={60pt,between origins},row
      sep={60pt,between origins},nodes={asymmetrical rectangle}] (s)
      {
        &|[name=ka]| \ker a &|[name=kb]| \ker b &|[name=kc]| \ker c \\
        %
        &|[name=A]| A' &|[name=B]| B' &|[name=C]| C' &|[name=01]| 0 \\
        %
        |[name=02]| 0 &|[name=A']| A &|[name=B']| B &|[name=C']| C \\
        %
        &|[name=ca]| \coker a &|[name=cb]| \coker b &|[name=cc]| \coker c \\
      };
      \draw[->] (ka) edge (A)
      (kb) edge (B)
      (kc) edge (C)
      (A) edge node {$f$} (B)
      (B) edge node {$g$} (C)
      (C) edge (01)
      (A) edge node {$a$} (A')
      (B) edge node {$b$} (B')
      (C) edge node {$c$} (C')
      (02) edge (A')
      (A') edge node {$f'$} (B')
      (B') edge node {$g'$} (C')
      (A') edge (ca)
      (B') edge (cb)
      (C') edge (cc)
      ;
      \draw[->,gray] (ka) edge (kb)
      (kb) edge (kc)
      (ca) edge (cb)
      (cb) edge (cc)
      ;
      \draw[->,gray,rounded corners] (kc) -| node[auto,text=black,pos=.7]
      {$d$} ($(01.east)+(.5,0)$) |- ($(B)!.35!(B')$) -|
      ($(02.west)+(-.5,0)$) |- (ca);
    \end{tikzpicture}
  \end{center}
\end{lemma*}

We also should use some matrices.
\begin{theorem*}
  Let $F_{n}$ be the $n$th Fibonacci number.
  Then we have
  \begin{equation*}
    \begin{bmatrix}
      F_{n+2} \\
      F_{n+1}
    \end{bmatrix} =
    \begin{bmatrix}
      1 & 1 \\
      1 & 0
    \end{bmatrix}
    \begin{bmatrix}
      F_{n+1} \\
      F_{n}
    \end{bmatrix}
  \end{equation*}
  and thus
  \begin{equation*}
    \begin{bmatrix}
      1 & 1 \\
      1 & 0
    \end{bmatrix}^{n} =
    \begin{bmatrix}
      F_{n+1} & F_{n} \\
      F_{n} & F_{n-1}
    \end{bmatrix}.
  \end{equation*}
\end{theorem*}

\section{Tables}
\LaTeX{} is also great at tables, especially in conjuction with the \code{booktabs} package.
See \cref{tab:example} for an example with multi-column labels and intra-column rules.
Don't go too crazy with these things, though---less is more when it comes to table layout!

\begin{table}[ph]
  \centering
  \begin{tabular}{l l l l l}
    \toprule
    & \multicolumn{2}{c}{Revenue} & \multicolumn{2}{c}{Expenses} \\
    \cmidrule(lr){2-3}
    \cmidrule(lr){4-5}
    District & Dec. & YTD & Dec. & YTD \\ \midrule
    North & 57 & 1024 & 12 & 590 \\
    South & 100 & 2032 & 94 & 1502 \\
    \bottomrule
  \end{tabular}
  \caption{An example of a fancy table}
  \label{tab:example}
\end{table}

% If you want to include appendices, just use the \appendix command
% and then make chapters as normal
\appendix
\chapter{Useful resources}
There's a huge community of \LaTeX{} authors and users on the Internet, who have developed an amazing array of resources to help you learn.

% Finally, we switch over to "back matter" mode for the bibliography
\backmatter
\printbibliography
\end{document}