% Template for Carleton comps papers
% Author: Andrew Gainer-Dewar, 20131
\documentclass[twoside ]{memoir}
\usepackage{cccomps}

% The Latin Modern font is a modernized replacement for the classic
% Computer Modern. Feel free to replace this with a different font package.
\usepackage{lmodern}

\title{Title of the paper}
\author{Author 1 \and Author 2 \and Author 3 \and Author 4}
\advisor{Advisor name}
\dept{Department}
\date{Date}

\begin{document}
% First, we go into "front matter" mode.
% Among other things, this gives us Roman page numbers.
\frontmatter

% We tell LaTeX to make a title page.
\maketitle

% Then we do some front-matter business.
\chapter{Dedication}
If you wish to write a dedication or acknowledgments, you can do so here.

\begin{abstract}
  Write an abstract here.
  Be sure to keep it under 300 words.
\end{abstract}

% A Table of Contents is important!
\tableofcontents

% Include the list of figures and list of tables only if you actually *have*
% figures and tables! (The * after each indicates that it should not be included
% in the table of contents.)
%\listoffigures*
%\listoftables*

% Next, we go into "main matter" mode.
% This resets the page numbers and uses Arabic numerals.
\mainmatter

\chapter{Title of a chapter}
Introductory matter for this section.

\section{Title of a section}
Something in this section.

\begin{theorem}
  Theorem statement
\end{theorem}

\section{Another title of a section}
Something for the next section.

% If you want to include appendices, just use the \appendix command
% and then make chapters as normal
\appendix
\chapter{An appendix}
Be sure to check out \cite{notsoshort} for more information!

% Finally, we switch over to "back matter" mode for the bibliography
\backmatter
\bibliographystyle{amsplain}
\bibliography{sources}
\end{document}